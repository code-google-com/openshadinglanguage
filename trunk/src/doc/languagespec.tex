\documentclass[11pt,letterpaper]{book}
\setlength{\oddsidemargin}{0.5in}
\setlength{\topmargin}{0in}
\setlength{\evensidemargin}{0.3in}
\setlength{\textwidth}{5.75in}
\setlength{\textheight}{8.5in}
%\setlength{\oddsidemargin}{1.25in}
%\setlength{\evensidemargin}{0.5in}

% don't do this \usepackage{times}    % Better fonts than Computer Modern
\renewcommand{\sfdefault}{phv}
\renewcommand{\rmdefault}{ptm}
% don't replace tt -- old is better \renewcommand{\ttdefault}{pcr}
%\usepackage{apalike}
\usepackage{pslatex}
\usepackage{techref}
\usepackage{epsfig}
\usepackage{verbatim}
\usepackage{moreverb}
\usepackage{graphicx}
\usepackage{xspace}
\usepackage{multicol}
\usepackage{color}
\usepackage{html}
\usepackage{version}
\usepackage{makeidx}
%\usepackage{showidx}
\usepackage[chapter]{algorithm}
\floatname{algorithm}{Listing}

\usepackage{syntax}

\usepackage{fancyhdr}
\pagestyle{fancy}
\fancyhead[LE,RO]{\bfseries\thepage}
\fancyhead[LO]{\bfseries\rightmark}
\fancyhead[RE]{\bfseries\leftmark}
\fancyfoot[C]{\bfseries Unnamed Shading Language Documentation}
\renewcommand{\footrulewidth}{1pt}


\def\langname{Unnamed Shading Language\xspace}
\def\product{{\sffamily Unnamed Shading Language}\xspace}
\def\versionnumber{0.1}
\def\productver{\product\ {\sffamily \versionnumber}\xspace}


\title{ 
{\Huge{\bf \product}
%\textregistered\ 
{\bf\sffamily \versionnumber} \medskip \\ \huge 
Language Specification
\\ \large (draft in progress) 
} \bigskip }
\author{Editor: Editor's Name \\
% email should go here
 \bigskip \\
}
\date{{\large Date: 16 October, 2008}}


%%%%%%%%%%%%%%%%%%%%%%%%%%%%%%%%%%%%%%%%%%%%%%%%%%%%%%%%%%%%%%%%%%%%%%%%%%
% Larry's favorite LaTeX macros for making technical books.  These have
% been refined for years, starting with SIGGRAPH course notes in the
% '90's, further refined for _Advanced RenderMan_.
%
% Please use or modify this at will.
%%%%%%%%%%%%%%%%%%%%%%%%%%%%%%%%%%%%%%%%%%%%%%%%%%%%%%%%%%%%%%%%%%%%%%%%%%

%
% Define typesetting commands for filenames and code
%
% Just like Advanced RenderMan -- all code in courier, keywords in text
%    courier but not bold.
\def\codefont{\ttfamily}	% font to use for code
\def\ce{\codefont\bfseries}	% emphasize something in code


%
% Define typesetting commands for filenames and code
%
\def\cf{\codefont}		% abbreviation for \codefont
\def\fn{\codefont}		% in-line filenames & unix commands
\def\kw{\codefont}      	% in-line keyword

\newcommand{\var}[1]{{\kw \emph{#1}}}  % variable
\newcommand{\qkw}[1]{{\kw "#1"}}       % quoted keyword



% Define some environments for easy typesetting of small amounts of
% code.  These are mostly just wrappers around verbatim, but the
% different varieties also change font sizes.
\newenvironment{code}{\small \verbatimtab}{\endverbatimtab}
\newenvironment{smallcode}{\small \renewcommand{\baselinestretch}{0.8} \verbatimtab}{\endverbatimtab \renewcommand{\baselinestretch}{1}}
\newenvironment{tinycode}{\footnotesize \renewcommand{\baselinestretch}{0.75} \verbatimtab}{\endverbatimtab \renewcommand{\baselinestretch}{1}}

\begin{htmlonly}
\renewenvironment{code}{\begin{verbatim}}{\end{verbatim}}
\newenvironment{smallcode}{\begin{verbatim}}{\end{verbatim}}
\newenvironment{tinycode}{\begin{verbatim}}{\end{verbatim}}
\end{htmlonly}

\newcommand{\includedcode}[1]{{\small \verbatimtabinput{#1}}}
\newcommand{\smallincludedcode}[1]{{\small \renewcommand{\baselinestretch}{0.8} \verbatimtabinput{#1} \renewcommand{\baselinestretch}{1}}}
\newcommand{\tinyincludedcode}[1]{{\footnotesize \renewcommand{\baselinestretch}{0.75} \verbatimtabinput{#1} \renewcommand{\baselinestretch}{1}}}





% Also create a hyphenation list, essentially just to guarantee that
% type names aren't hyphenated
%\hyphenation{Attribute}

% Handy for parameter lists
\def\pl{\emph{parameterlist}\xspace}
\def\epl{\emph{...parameterlist...}\xspace}
\hyphenation{parameterlist}




%%%%%%%%%%%%%%%%%%%%%%%%%%%%%%%%%%%%%%%%%%%%%%%%%%%%%%%%%%%%%%%%%%%%%%%%%%%%%
%
% Defs of variables and functions that we frequently refer to in
% text.
%

\def\ParamType{{\kw ParamType}\xspace}

%%%%%%%%%%%%%%%%%%%%%%%%%%%%%%%%%%%%%%%%%%%%%%%%%%%%%%%%%%%%%%%%%%%%%%%%%%%%




%begin{latexonly}
\newenvironment{apilist}{\begin{list}{}{\medskip \item[]}}{\end{list}}
\newcommand{\apiitem}[1]{\vspace{12pt} \noindent {\bf\tt #1} \vspace{-10pt}\begin{apilist}\nopagebreak[4]}
\newcommand{\apiend}{\end{apilist}\medskip\pagebreak[2]}
\def\bigspc{\makebox[72pt]{}}
\def\spc{\makebox[24pt]{}}
\def\halfspc{\makebox[12pt]{}}
\def\neghalfspc{\hspace{-12pt}}
\def\negspc{\hspace{-24pt}}
\def\chapwidthbegin{}
\def\chapwidthend{}
%end{latexonly}


\begin{htmlonly}
\newcommand{\apiitem}[1]{\medskip \noindent {\bf #1} \begin{quote}}
\newcommand{\apiend}{\end{quote}}
\def\halfspc{\begin{rawhtml} &nbsp; &nbsp; \end{rawhtml}}
\def\spc{\halfspc\halfspc}
\pagecolor[named]{White}
\def\chapwidthbegin{\begin{rawhtml}<p><table cellspacing=1><tr><td width=550>\end{rawhtml}}
\def\chapwidthend{\begin{rawhtml}</td></tr></table>\end{rawhtml}}
\end{htmlonly}


\newcommand{\apibinding}[3]{\apiitem{#1\\[1ex]#2\\[1ex]#3}}

\newcommand{\CPPBINDING}[1]{\par {\small C++ BINDING:}\par {\spc \codefont #1}}
\newcommand{\PARAMETERS}{\par {\small PARAMETERS:} \par}
\newcommand{\EXAMPLE}{\par {\small EXAMPLE:} \par}
\newcommand{\EXAMPLES}{\par {\small EXAMPLES:} \par}
\newcommand{\SEEALSO}{\par \hspace{-20pt} See Also: \par}



% The \begin{algorithm} \end{algorithm} macros (in algorithm.sty) are
% great for code that can fit all on one page.  But when it can't, use
% these macros.  The first parameter is the caption, the second is the
% label name.
\newcommand{\longalgorithmbegin}[2]{\noindent\hrulefill \\
  \refstepcounter{algorithm}
  \noindent {\bf Listing \arabic{chapter}.\arabic{algorithm}}: #1 \label{#2} \\
  \addcontentsline{loa}{algorithm}{\numberline {\arabic{algorithm}} #1}
  \noindent\hrulefill
}
\newcommand{\longalgorithmend}{\noindent\hrulefill \\}


\def\NEW{\marginpar[\medskip\hfill~\fbox{\sffamily \Huge NEW!}~]{\medskip~\fbox{\sffamily \Huge NEW!}~}}
\newcommand{\NEWdown}[1]{\marginpar[\vspace{#1}\hfill\fbox{\sffamily \Huge NEW!}]{\vspace{#1}\fbox{\sffamily \Huge NEW!}}}
\def\DEPRECATED{\marginpar[\medskip\hfill~\fbox{\sffamily \Large Deprecated}]{\medskip~\fbox{\sffamily \Large Deprecated}}}
\newcommand{\DEPRECATEDdown}[1]{\marginpar{\vspace{#1}\fbox{\sffamily \Large Deprecated}}}
\def\CHANGED{\marginpar[\medskip\hfill~\fbox{\sffamily \huge CHANGED!}~]{\medskip~\fbox{\sffamily \huge CHANGED!}~}}
\def\ENHANCED{\marginpar[\medskip\hfill~\fbox{\sffamily \huge ENHANCED}~]{\medskip~\fbox{\sffamily \huge ENHANCED}~}}
\def\QUESTION{\marginpar[\medskip\hfill~\fbox{\sffamily \Huge ?}~~~~]{\medskip~\fbox{\sffamily \Huge ?}~~~~}}


\newcommand{\indexapi}[1]{\index{#1@\tt#1\rm}}



\newenvironment{annotate}{\medskip\sffamily\em\noindent}{\medskip}
%\newenvironment{annotate}{\begin{comment}}{\end{comment}}


\def\color{{\cf color}\xspace}
\def\float{{\cf float}\xspace}
\def\inttype{{\cf int}\xspace}
\def\matrix{{\cf matrix}\xspace}
\def\normal{{\cf normal}\xspace}
\def\point{{\cf point}\xspace}
\def\vector{{\cf vector}\xspace}
\def\P{{\cf P}\xspace}
\def\N{{\cf N}\xspace}
\def\I{{\cf I}\xspace}
\def\currentspace{{\cf "current"} space\xspace}
\def\commonspace{{\cf "common"} space\xspace}
\def\shaderspace{{\cf "shader"} space\xspace}
\def\worldspace{{\cf "world"} space\xspace}
\def\cameraspace{{\cf "camera"} space\xspace}
\def\objectspace{{\cf "object"} space\xspace}
\def\rgbspace{{\cf "rgb"} space\xspace}
\def\noise{{\cf noise()}\xspace}
\def\snoise{{\cf snoise()}\xspace}
\def\pnoise{{\cf pnoise()}\xspace}
\def\psnoise{{\cf psnoise()}\xspace}
\def\cellnoise{{\cf cellnoise()}\xspace}





\makeindex

\begin{document}
\frontmatter

\maketitle

%\include{speccopyr}

\vspace*{2in}




\setcounter{tocdepth}{1}
\tableofcontents

\mainmatter

%\part{Part name}

%\include{blah}


\chapter{Basic Structure}
\label{chap:name}

shader types

lexical issues

identifiers

comments

preprocessor


\chapter{Data types, variables, and parameters}

\section{Data types}

\langname provides several built-in data types for performing
computations inside your shader:
\smallskip

\noindent\begin{tabular}{|p{1in}|p{4.5in}|}
\hline
{\cf int}	& Integer data \\
\hline
{\cf float}	& Scalar floating-point data (numbers) \\
\hline
{\cf point
vector
normal}		& Three-dimensional positions, directions, and surface
		  orientations \\
\hline
{\cf color}	& Spectral reflectivities and light energy values \\
\hline
{\cf matrix}	& $4 \times 4$ transformation matrices \\
\hline
{\cf string}	& Character strings (such as filenames) \\
\hline
\end{tabular}


\subsection{{\cf int}}

The basic type for discrete numeric values is {\cf int}.  The size of
the {\cf int} type is renderer-dependent, but is guaranteed to be at
least 32 bits.

Integer constants are constructed the same way as in C.  The following
are examples of {\cf int} constants: {\cf 1}, {\cf -32}, etc.

Unlike C, no unsigned, bool, char, short, or long types are supplied.
This is to simplify the process of writing shaders (as well as
implementing shading systems).

The following operators may be used with \inttype values (in order of
decreasing precedence, with each box holding operators of the same
precedence):

\smallskip

\noindent \begin{tabular}{|p{1in}|p{0.5in}|p{3.75in}|}
operation & result & ~ \\
\hline
\inttype\ {\ce ++} & \inttype & post-increment by 1 \\[0.5ex]
\inttype\ {\ce --} & \inttype & post-decrement by 1 \\[0.5ex]
\hline
{\ce ++} \inttype & \inttype & pre-increment by 1 \\[0.5ex]
{\ce --} \inttype & \inttype & pre-decrement by 1 \\[0.5ex]
{\cf {\bfseries -} int} & \inttype & unary negation \\[0.5ex]
\verb|~| {\cf int} & \inttype & bitwise complement (1 and 0 bits flipped) \\[0.5ex]
{\cf {\bfseries !} int} & \inttype & boolean `not' (1 if operand is zero, otherwise 0) \\[0.5ex]
\hline
{\cf int {\bfseries *} int} & \inttype & multiplication \\[0.5ex]
{\cf int {\bfseries /} int} & \inttype & division \\[0.5ex]
{\cf int {\bfseries \%} int} & \inttype & modulus \\[0.5ex]
\hline
{\cf int {\bfseries +} int} & \inttype & addition \\[0.5ex]
{\cf int {\bfseries -} int} & \inttype & subtraction \\[0.5ex]
\hline
{\cf int {\bfseries <<} int} & \inttype & shift left \\[0.5ex]
{\cf int {\bfseries >>} int} & \inttype & shift right \\[0.5ex]
\hline
{\cf int {\bfseries <} int} & \inttype & 1 if the first value is less
than the second, else 0 \\[0.5ex]
{\cf int {\bfseries <=} int} & \inttype & 1 if the first value is less
or equal to the second, else 0 \\[0.5ex]
{\cf int {\bfseries >} int} & \inttype &  1 if the first value is
greater than the second, else 0 \\[0.5ex]
{\cf int {\bfseries >=} int} & \inttype & 1 if the first value is
greater than or equal to the second, else 0 \\[0.5ex]
\hline
{\cf int {\bfseries ==} int} & \inttype & 1 if the two values are equal,
else 0 \\[0.5ex]
{\cf int {\bfseries !=} int} & \inttype & 1 if the two values are different,
else 0 \\[0.5ex]
\hline
{\cf int {\bfseries \&} int} & \inttype & bitwise and \\[0.5ex]
\hline
{\cf int {\bfseries \verb|^|} int} & \inttype & bitwise exclusive or \\[0.5ex]
\hline
{\cf int {\bfseries |} int} & \inttype & bitwise or \\[0.5ex]
\hline
{\cf int {\bfseries \&\&} int} & \inttype & boolean and (1 if both operands are
nonzero, otherwise 0) \\[0.5ex]
\hline
{\cf int {\bfseries ||} int} & \inttype & boolean or (1 if either operand is
nonzero, otherwise 0) \\[0.5ex]
\hline
\end{tabular}



\begin{annotate}
Make sure to rearrange the table in order of operator precedence.
\end{annotate}

\subsection{{\cf float}}

The basic type for scalar floating-point numeric values is \float.  The
size of the {\cf float} type is renderer-dependent, but is guaranteed to
be at least IEEE 32-bit float (the standard C \float data type).
Individual renderer implementations may choose to implement \float with
even more precision (such as using the C {\cf double} as the underlying
representation).

Floating-point constants are constructed the same way as in C.
The following are examples of {\cf float} constants:  {\cf 1.0},
{\cf 2.48}, {\cf -4.3e2}.

The following operators may be used with \float values (in order of
decreasing precedence, with each box holding operators of the same
precedence):

\smallskip

\noindent \begin{tabular}{|p{1.25in}|p{0.5in}|p{3.5in}|}
operation & result & ~ \\
\hline
\float\ {\ce ++} & \float & post-increment by 1 \\[0.5ex]
\float\ {\ce --} & \float & post-decrement by 1 \\[0.5ex]
\hline
{\ce ++} \float & \float & pre-increment by 1 \\[0.5ex]
{\ce --} \float & \float & pre-decrement by 1 \\[0.5ex]
{\cf {\bfseries -} float} & \float & unary negation \\[0.5ex]
\hline
{\cf float {\bfseries *} float} & \float & multiplication \\[0.5ex]
{\cf float {\bfseries /} float} & \float & division \\[0.5ex]
\hline
{\cf float {\bfseries +} float} & \float & addition \\[0.5ex]
{\cf float {\bfseries -} float} & \float & subtraction \\[0.5ex]
\hline
{\cf float {\bfseries <} float} & \inttype & 1 if the first value is less
than the second, else 0 \\[0.5ex]
{\cf float {\bfseries <=} float} & \inttype & 1 if the first value is less
or equal to the second, else 0 \\[0.5ex]
{\cf float {\bfseries >} float} & \inttype &  1 if the first value is
greater than the second, else 0 \\[0.5ex]
{\cf float {\bfseries >=} float} & \inttype & 1 if the first value is
greater than or equal to the second, else 0 \\[0.5ex]
\hline
{\cf float {\bfseries ==} float} & \inttype & 1 if the two values are equal,
else 0 \\[0.5ex]
{\cf float {\bfseries !=} float} & \inttype & 1 if the two values are different,
else 0 \\[0.5ex]
\hline
\end{tabular}

All of the binary operators may combine a \float with an \inttype,
by promoting the \inttype to a \float.


\subsection{{\cf color}}

The \color type is used to represent 3-component (RGB) spectral
reflectivities and light energies.  You can assemble a
color out of three floats, either representing an RGB triple or some
other color space known to the renderer, as well as from a single
float (replicated for all three channels).  Following are some examples:

\begin{code}
    color (0, 0, 0)              // black
    color ("rgb", .75, .5, .5)   // pinkish
    color ("hsv", .2, .5, .63)   // specify in "hsv" space
    color (0.5)                  // same as color (0.5, 0.5, 0.5)
\end{code}

All these expressions above return colors in \rgbspace.  Even
the third example returns a color in \rgbspace\ --- specifically, the
RGB value of the color that is equivalent to hue 0.2, saturation 0.5,
and value 0.63.  In other words, when assembling a color from components
given relative to a specific color space in this manner, there is an
implied transformation to \rgbspace.  Table~\ref{tab:colorspacenames}
lists the built-in color spaces.

\begin{table}[htbp]
\caption{Names of color spaces.}\label{tab:colorspacenames}
\begin{tabular}{|p{0.7in}|p{4.7in}|}
\hline
{\cf "rgb"} & The coordinate system that all colors start out in, and
in which the renderer expects to find colors that are set by
your shader (such as {\cf C}, {\cf opacity}, and {\cf Cl}).   \\
\hline 
{\cf "hsv"} & hue, saturation, and value. \\
\hline 
{\cf "hsl"} & hue, saturation, and lightness. \\
\hline 
{\cf "YIQ"} & the color space used for the NTSC television standard. \\
\hline 
{\cf "xyz"} & CIE \emph{XYZ} coordinates. \\
\hline 
{\cf "xyY"} & CIE \emph{xyY} coordinates. \\
\hline 
\end{tabular}
\end{table}

Colors may be assigned another color or a \float value (which sets
all three components to the value).  For example:

\begin{code}
    color C;
    C = color (0, 0.3, 0.3);
    C = 0.5;                    // same as C = color (0.5, 0.5, 0.5)
\end{code}

Colors can have their individual components examined and set using the
{\cf []} array access notation.  For example:

\begin{code}
    color C;
    float g = C[1];   // get the green component
    C[0] = 0.5;       // set the red component
\end{code}

\noindent Components 0, 1, and 2 are red, green, and blue, respectively.
It is an error to access a color component with an index outside the
$[0...2]$ range.

The following operators may be used with \color values (in order of
decreasing precedence, with each box holding operators of the same
precedence):

\medskip

\noindent \begin{tabular}{|p{1.5in}|p{0.5in}|p{3.25in}|}
operation & result & ~ \\
\hline
\color\ {\ce [} \inttype {\ce ]} & \float & component access \\[0.5ex]
\hline
{\ce -} \color & \color & unary negation \\[0.5ex]
\hline
\color\ {\ce *} \color & \color & component-wise multiplication \\[0.5ex]
\color\ {\ce *} \float & \color & scaling \\[0.5ex]
\float\ {\ce *} \color & \color & scaling \\[0.5ex]
\color\ {\ce /} \color & \color & component-wise division \\[0.5ex]
\color\ {\ce /} \float & \color & scaling \\[0.5ex]
\float\ {\ce /} \color & \color & scaling \\[0.5ex]
\hline
\color\ {\ce +} \color & \color & component-wise addition \\[0.5ex]
\color\ {\ce -} \color & \color & component-wise subtraction \\[0.5ex]
\hline
\color\ {\ce ==} \color & \inttype & 1 if the two values are equal,
else 0 \\[0.5ex]
\color\ {\ce !=} \color & \inttype & 1 if the two values are different,
else 0 \\[0.5ex]
\hline
\end{tabular}

All of the binary operators may combine a scalar value (\float or
\inttype) with a \color, treating the scalar if it were a \color with
three identical components.


\subsection{Point-like types: {\cf point}, {\cf vector}, {\cf normal}}

Points, vectors, and normals are similar data types with identical
structures but subtly different semantics.  We will frequently refer to
them collectively as the ``point-like'' data types when making
statements that apply to all three types.

A \point is a position in 3D space.  A \vector has a length and
direction, but does not exist in a particular location.  A \normal is a
special type of vector that is \emph{perpendicular} to a surface, and
thus describes the surface's orientation.  Such a perpendicular vector
uses different transformation rules than ordinary vectors, as we will
describe below.

All of these point-like types are internally represented by three
floating-point numbers that uniquely describe a position or
direction relative to the three axes of some coordinate system.  

All points, vectors, and normals are described relative to some
coordinate system.  All data provided to a shader (surface information,
graphics state, parameters, and vertex data) are relative to one
particular coordinate system that we call the {\cf "common"} coordinate
system.  The {\cf "common"} coordinate system is one that is convenient
for the renderer's shading calculations.
\label{sect:commonspace}

You can ``assemble'' a point-like type out of three floats using a
constructor:

\begin{code}
        point (0, 2.3, 1)
        vector (a, b, c)
        normal (0, 0, 1)
\end{code}

\noindent These expressions are interpreted as a point, vector, and normal
whose three components are the floats given, relative to \commonspace\ .

As with colors, you may also specify the coordinates relative to some other
coordinate system:

\begin{code}
    Q = point ("object", 0, 0, 0);
\end{code}

This example assigns to {\cf Q} the point at the origin of
\objectspace.  However, this statement does \emph{not} set the
components of {\cf Q} to (0,0,0)!  Rather, {\cf Q} will contain the
\commonspace\ coordinates of the point that is at the same location as
the origin of \objectspace.  In other words, the point constructor that
specifies a space name implicitly specifies a transformation
to \commonspace.  This type of constructor also can be used for vectors
and normals.

The choice of \commonspace is renderer-dependent, though will usually
be equivalent to either \cameraspace or \worldspace.

Some computations may be easier in a coordinate system other than
\currentspace.  For example, it is much more convenient to apply a
``solid texture'' to a moving object in its \objectspace than in
\currentspace.  For these reasons, SL provides built-in functions that
allow you to transform points among different coordinate systems.  The
built-in functions {\cf transform, vtransform, ntransform} can be used
to transform points, vectors, and normals, respectively, from one
coordinate system to another (see Section~\ref{chap:stdlibrary}).  Note,
however, that \langname does not keep track of which point variables are
in which coordinate systems.  It is the responsibility of the shader
programmer to keep track of this and ensure that, for example, lighting
computations are performed using quantities in \commonspace.

Be very careful to use the right transformation routines for the right
point-like types.  Transforming with the wrong matrix math will
introduce subtle and difficult-to-fix errors in your code.  Therefore,
it is important to always use {\cf transform} for points, {\cf
  vtransform} for vectors, and {\cf ntransform} for normals.

Several coordinate systems are predefined by name, listed in
Table~\ref{tab:spacenames}.  Additionally, a renderer will probably
allow for additional coordinate systems to be named in the scene
description, and these names may also be referenced inside your shader
to designate transformations.

\begin{table}[htbp]
\caption{Names of predeclared geometric spaces.\label{tab:spacenames}}
\begin{tabular}{|p{0.7in}|p{4.7in}|}
\hline
{\cf "common"} & The coordinate system that all spatial values start out in and
the one in which all lighting calculations are carried out.  Note that
the choice of {\cf "common"} space may be different on each renderer. \\
\hline 
{\cf "object"} & The local coordinate system of the graphics primitive (sphere,
patch, etc.) that we are shading. \\
\hline 
{\cf "shader"} & The local coordinate system active at the time that the shader
was instanced. \\
\hline 
{\cf "world"} & The world coordinate system designated in the scene. \\
\hline 
{\cf "camera"} & The coordinate system with its origin at the center of
the camera lens, $x$-axis pointing right, $y$-axis pointing up, and
$z$-axis pointing into the screen. \\
\hline 
{\cf "screen"} & The coordinate system of the camera's image plane
(after perspective transformation, if any).  Coordinate (0,0) of {\cf
"screen"} space is looking along the $z$-axis of \cameraspace. \\
\hline 
{\cf "raster"} & 2D pixel coordinates, with (0,0) as the upper-left
corner of the image and (xres, yres) as the lower-right corner. \\
\hline 
{\cf "NDC"} & 2D Normalized Device Coordinates --- like raster space, but
normalized so that $x$ and $y$ both run from 0 to 1 across the whole
image, with (0,0) being at the upper left of the image, and (1,1) being
at the lower right. \\
\hline 
\end{tabular}
\end{table}

Point types can have their individual components examined and set using
the {\cf []} array access notation.  For example:

\begin{code}
    point P;
    float y = C[1];   // get the y component
    C[0] = 0.5;       // set the x component
\end{code}

\noindent Components 0, 1, and 2 are $x$, $y$, and $z$, respectively.
It is an error to access a point component with an index outside the
$[0...2]$ range.

The following operators may be used with point-like values (in order of
decreasing precedence, with each box holding operators of the same
precedence):

\medskip

\noindent \begin{tabular}{|p{1.5in}|p{0.5in}|p{3.25in}|}
operation & result & ~ \\
\hline
\emph{ptype}\ {\ce [} \inttype {\ce ]} & \float & component access \\[0.5ex]
\hline
{\ce -} \emph{ptype} & \vector & component-wise unary negation \\[0.5ex]
\hline
\emph{ptype}\ {\ce *} \emph{ptype} & \emph{ptype} & component-wise multiplication \\[0.5ex]
\float\ {\ce *} \emph{ptype} & \emph{ptype} & scaling of all components \\[0.5ex]
\emph{ptype}\ {\ce *} \float & \emph{ptype} & scaling of all components \\[0.5ex]
\emph{ptype}\ {\ce /} \emph{ptype} & \emph{ptype} & component-wise division \\[0.5ex]
\emph{ptype}\ {\ce /} \float & \emph{ptype} & division of all components \\[0.5ex]
\float\ {\ce /} \emph{ptype} & \emph{ptype} & division by all components \\[0.5ex]
\hline
\emph{ptype}\ {\ce +} \emph{ptype} & \emph{ptype} & component-wise addition \\[0.5ex]
\emph{ptype}\ {\ce -} \emph{ptype} & \vector & component-wise subtraction \\[0.5ex]
\hline
\emph{ptype}\ {\ce ==} \emph{ptype} & \inttype & 1 if the two values are equal,
else 0 \\[0.5ex]
\emph{ptype}\ {\ce !=} \emph{ptype} & \inttype & 1 if the two values are different,
else 0 \\[0.5ex]
\hline
\end{tabular}

\smallskip

\noindent The generic \emph{ptype} is listed in places where any
of \point, \vector, or \normal may be used.

All of the binary operators may combine a scalar value (\float or
\inttype) with a point-like type, treating the scalar if it were
point-like with three identical components.


\subsection{{\cf matrix}}

\langname has a \matrix type that represents the transformation matrix
required to transform points and vectors between one coordinate system
and another.  Matrices are represented internally by 16 floats (a $4
\times 4$ homogeneous transformation matrix).

A \matrix can be constructed from a single float or 16 floats.  For
example:

\begin{code}
    matrix zero = 0;   // makes a matrix with all 0 components
    matrix ident = 1;  // makes the identity matrix

    // Construct a matrix from 16 floats
    matrix m = matrix (m00, m01, m02, m03, m10, m11, m12, m13, 
                       m20, m21, m22, m23, m30, m31, m32, m33);
\end{code}

\noindent Assigning a single floating-point number $x$ to a matrix will result
in a matrix with diagonal components all being $x$ and other
components being zero (i.e., $x$ times the identity matrix).
Constructing a matrix with 16 floats will create the matrix whose
components are those floats, in row-major order.  

Similar to point-like types, a {\cf matrix} may be constructed in
reference to a named space:

\begin{code}
    // Construct matrices relative to something other than "common"
    matrix q = matrix ("shader", 1);
    matrix m = matrix ("world", m00, m01, m02, m03, m10, m11, m12, m13, 
                               m20, m21, m22, m23, m30, m31, m32, m33);
\end{code}

\noindent The first form creates the matrix that transforms points from
\currentspace to \shaderspace.  Transforming points by this matrix
is identical to calling {\cf transform("shader",...)}.
The second form prepends the current-to-world transformation matrix
onto the $4 \times 4$ matrix with components $m_{0,0} ... m_{3,3}$.
Note that although we have used {\cf "shader"} and {\cf "world"}
space in our examples, any named space is acceptable.

Matrix variables can be tested for equality and inequality with the
{\cf ==} and {\cf !=} boolean operators.  Also, the {\cf *} operator
between matrices denotes matrix multiplication, while {\cf m1 / m2}
denotes multiplying {\cf m1} by the inverse of matrix {\cf m2}.  Thus,
a matrix can be inverted by writing {\cf 1/m}.  In addition, some
functions will accept matrix variables as arguments, as described in
Section~\ref{chap:stdlibrary}.

Individual compoents of a matrix variable may be set or accessed
using array notation, for example,

\begin{code}
    matrix M;
    float x = M[row][col];
    M[row][col] = 1;
\end{code}

Valid component indices are integers on $[0...3]$.  It is an error to
access a matrix component with either a row or column outside this
range.

The following operators may be used with matrices (in order of
decreasing precedence, with each box holding operators of the same
precedence):

\medskip
\noindent \begin{tabular}{|p{1.5in}|p{0.5in}|p{3.25in}|}
operation & result & ~ \\
\hline
\matrix\ {\ce [} \inttype {\ce ][} \inttype {\ce ]} & \float & component
access (row, column) \\[0.5ex]
\hline
{\ce -} \matrix & \matrix & unary negation \\[0.5ex]
\hline
\matrix\ {\ce *} \matrix & \matrix & matrix multiplication \\[0.5ex]
\matrix\ {\ce *} \float & \matrix & component-wise scaling \\[0.5ex]
\float\ {\ce *} \matrix & \matrix & component-wise scaling \\[0.5ex]
\matrix\ {\ce /} \matrix & \matrix & multiply the first matrix by the
\emph{inverse} of the second \\[0.5ex]
\matrix\ {\ce /} \float & \matrix & component-wise division \\[0.5ex]
\float\ {\ce /} \matrix & \matrix & multiply the \float by the
\emph{inverse} of the matrix \\[0.5ex]
\hline
\matrix\ {\ce +} \matrix & \matrix & component-wise addition \\[0.5ex]
\matrix\ {\ce -} \matrix & \matrix & component-wise subtraction \\[0.5ex]
\hline
\matrix\ {\ce ==} \matrix & \inttype & 1 if the two values are equal,
else 0 \\[0.5ex]
\matrix\ {\ce !=} \matrix & \inttype & 1 if the two values are different,
else 0 \\[0.5ex]
\hline
\end{tabular}


\subsection{{\cf string}}

The {\cf string} type may hold character strings.  The main application
of strings is to provide the names of files where textures may be
found.  Strings can be compared using {\cf ==} and {\cf !=}.

String constants are denoted by surrounding the characters with double
quotes, as in \qkw{I am a string literal}.  As in C programs, string
literals may contain escape sequences such as \verb|\n| (newline),
\verb|\r| (carriage return), \verb|\t| (tab), \verb|\"| (double quote),
\verb|\\| (backslash).

\section{Variables, arrays, and structures}

\subsection{Variable declarations}

The syntax for declaring a variable in \langname is:

\vspace{12pt}
\spc \emph{type} \emph{variablename} 

\spc \emph{type} \emph{variablename} [ = \emph{initializer}]
\vspace{12pt}

\noindent where

\begin{itemize}
\item \emph{type} is one of the basic data types, described earlier.
\item \emph{variablename} is the name of the variable you are declaring.
\item If you wish to give your variable an initial value, you may do so
by assigning an \emph{initializer}.
\end{itemize}

\noindent Some examples of variable declarations are

\begin{code}
    float a;          // Declare; current value is undefined
    float b = 1;      // Declare and assign a constant initializer
    float c = a*b;    // Computed initializer
\end{code}

\subsection{Arrays}

Arrays are also supported, declared as follows:

\vspace{12pt}
\spc \emph{type} \emph{variablename}
 {\cf [} \emph{arraylen} {\cf ]}

\spc \emph{type} \emph{variablename}
 {\cf [} \emph{arraylen} {\cf ]} = {\cf \{ }
 \emph{init0}{\cf ,} \emph{init1} ... {\cf \}}
\vspace{12pt}

\noindent Arrays in \langname must have a constant length; they may not
be dynamically sized, and only 1D arrays are allowed.  Other than
that, however, the syntax of array usage in \langname is largely
similar to C.  Some examples of array variable declarations are:

\begin{code}
    float d[10];                       // Declare an uninitilized array
    float c[3] = { 0.1, 0.2, 3.14 };   // Initialize the array
\end{code}

\subsection{Structures}

Structures are used to group several fields of potentially different
types into a single object that can be referred to by name.  The syntax
for declaring a structure type is:

\vspace{12pt}
\spc {\cf struct} \emph{structname} {\cf \{} 

\spc\spc \emph{type1} \emph {fieldname1} {\cf ;}

\spc\spc ...

\spc\spc \emph{typeN} \emph {fieldnameN} {\cf ;}

\spc {\cf \} ;}
\vspace{12pt}

You may then use the structure type name to declare structure variables
as you would for any of the built-in types:

\vspace{12pt}
\spc \emph{structname} \emph{variablename} {\cf ;}

\spc \emph{structname} \emph{variablename} {\cf = \{ }
\emph{initializer1} {\cf ,} ... \emph{initializerN} {\cf \} ;}
\vspace{12pt}

If initializers are supplied, each field of the structure will be
initialized with the initializer in the corresponding position, which
is expected to be of the appropriate type.

Structure elements are accessed in the same way as other C-like
languages, using the `dot' operator: 

\vspace{12pt}
\spc \emph{variablename}{\cf .} \emph{fieldname}
\vspace{12pt}

Examples of declaration and use of structures:

\begin{code}
    struct ray {
        point pos;
        vector dir;
        float maxhitdist;
    };

    ray r;   // Declare a structure
    ray s = { point(0,0,0), vector(0,0,1), 1e6 };  // declare and initialize
    r.pos = point (1, 0, 0);  // Assign to one field
\end{code}

It is permitted to have a structure field that is an array, as well as
to have an array of structures.  But it is not permitted for one
structure to have a field that is another structure.  For example:

\begin{code}
    struct A {
        color a;
        float b[4];   // struct may contain an array
    };

    A c;                  // Declare a structure
    A d[5];               // Array of structures
    color e = d[0].a;     // Field of one element of array of struct
    d[2].b[4] = 0.25;     // Element of a field of a struct in an array
\end{code}

\begin{annotate}
\QUESTION Is this restriction (no ``structure recrusion'') acceptable?
It certainly makes the implementation of the language 

\end{annotate

\section{Shader parameters}

    how shader parameters get their values

shader metadata

\chapter{Statements and control flow}

language syntax

variable declarations and assignments

function calls

control flow (conditionals, while, do, for)

lighting

    emit/illuminate

    lights/illuminance

trace, the hard way

function definitions

    return

scoping

expressions

global variables

\chapter{Standard Library Functions}
\label{chap:stdlibrary}

\def\floatcolorpoint{The \emph{type} may be any of \float, \color,
  \point, \vector, or \normal.  For \color and \point-like types, the
  computations are performed component-by-component (separately for $x$,
  $y$, and $z$).\xspace}

\section{Basic math functions}
\label{sec:stdlib:math}

\subsection{Mathematical constants}

\langname defines several mathematical constants:

\medskip

\begin{tabular}{p{1in} p{3in}}
{\cf M\_PI}       & $\pi$ \\ \\
{\cf M\_PI\_2}    & $\pi/2$ \\ \\
{\cf M\_PI\_4}    & $\pi/4$ \\ \\
{\cf M\_E}        & $e$ \\ \\
{\cf M\_LN2}      & $\ln 2$ \\ \\
{\cf M\_LN10}     & $\ln 10$ \\ \\
{\cf M\_SQRT2}    & $\sqrt{2}$ \\ \\
{\cf M\_SQRT1\_2} & $\sqrt{1/2}$  \\ \\
\end{tabular}

\subsection{Mathematical functions}

Most of these functions operate on a generic \emph{type} that my be any
of \float, \color, \point, \vector, or \normal.  For \color and
\point-like types, the computations are performed component-by-component
(separately for $x$, $y$, and $z$).

\medskip

\apiitem{\emph{type} {\ce radians} (\emph{type} deg) \\
\emph{type} {\ce degrees} (\emph{type} rad)}
\indexapi{radians()} \indexapi{degrees()}
Convert degrees to radians or radians to degrees.
\apiend


\apiitem{\emph{type} {\ce cos} (\emph{type} x) \\
\emph{type} {\ce sin} (\emph{type} x) \\
\emph{type} {\ce tan} (\emph{type} x) }
\indexapi{cos()} \indexapi{sin()} \indexapi{tan()}
Computes the cosine, sine, and tangent of $x$ (measured in radians).
\apiend

\apiitem{\emph{type} {\ce acos} (\emph{type} x) \\
\emph{type} {\ce asin} (\emph{type} y) \\
\emph{type} {\ce atan} (\emph{type} y\_over\_x) \\
\emph{type} {\ce atan2} (\emph{type} y, \emph{type} x)}
\indexapi{acos()} \indexapi{asin()} \indexapi{atan()} \indexapi{atan2()}
Compute the principal value of the arc cosine, arc sine, and arc
For {\cf acos()} and {\cf asin()}, the value of the argument
will first be clamped to $[-1,1]$ to avoid invalid domain.

For {\cf acos()}, the result will always be in the range of $[0, \pi]$,
and for {\cf asin()} and {\cf atan()}, the result will always be in the
range of $[-\pi/2, \pi/2]$.  For {\cf atan2()}, the signs of both
arguments are used to determine the quadrant of the return value.
\apiend

\apiitem{\emph{type} {\ce cosh} (\emph{type} x) \\
\emph{type} {\ce sinh} (\emph{type} x) \\
\emph{type} {\ce tanh} (\emph{type} x) }
\indexapi{cosh()} \indexapi{sinh()} \indexapi{tanh()}
Computes the hyperbolic cosine, sine, and tangent of $x$ (measured in radians).
\apiend

\apiitem{\emph{type} {\ce pow} (\emph{type} x, \emph{type} y)}
\indexapi{pow()}
Computes $x^y$.  This function will return 0 for ``undefined''
operations, such as {\cf pow(-1,0.5)}.
\apiend

\apiitem{\emph{type} {\ce exp} (\emph{type} x) \\
\emph{type} {\ce exp2} (\emph{type} x) \\
\emph{type} {\ce expm1} (\emph{type} x)}
\indexapi{exp()} \indexapi{exp2()} \indexapi{expm1()}
Computes $e^x$, $2^x$, and $e^x-1$, respectively.  Note that 
{\cf expm1(x)} is accurate even for very small values of $x$.
\apiend

\apiitem{\emph{type} {\ce log} (\emph{type} x) \\
\emph{type} {\ce log2} (\emph{type} x) \\
\emph{type} {\ce log10} (\emph{type} x) \\
\emph{type} {\ce log} (\emph{type} x, \emph{type} b)}
\indexapi{log()} \indexapi{log2()} \indexapi{log10()}
Computes the logarithm of $x$ in base $e$, 2, 10, or arbitrary base $b$,
respectively.
\apiend

\apiitem{\emph{type} {\ce sqrt} (\emph{type} x) \\
\emph{type} {\ce inversesqrt} (\emph{type} x)}
\indexapi{sqrt()} \indexapi{inversesqrt()}
Computes $\sqrt{x}$ and $1/\sqrt{x}$.  Returns 0 if $x<0$.
\apiend

\apiitem{\emph{type} {\ce hypot} (\emph{type} x, \emph{type} y) \\
\emph{type} {\ce hypot} (\emph{type} x, \emph{type} y, \emph{type} x)}
\indexapi{hypot()}
Computes $\sqrt{x^2+y^2}$ and $\sqrt{x^2+y^2+z^z}$, respectively.
\apiend

\apiitem{\emph{type} {\ce abs} (\emph{type} x) \\
\emph{type} {\ce fabs} (\emph{type} x)} 
\indexapi{abs()} \indexapi{fabs()}
Absolute value of $x$.  (The two functions are synonyms.)
\apiend

\begin{annotate}
Should we fully return to C conventions, with abs() being for integers
and fabs being for float?  Or should abs() simply be overloaded, like
you'd do in C++ if we were starting from scratch today?
\end{annotate}

\apiitem{\emph{type} {\ce sign} (\emph{type} x)}
\indexapi{sign()}
Returns 1 if $x>0$, -1 if $x<0$, 0 if $x=0$.
\apiend

\apiitem{\emph{type} {\ce floor} (float x) \\
\emph{type} {\ce ceil} (\emph{type} x) \\
\emph{type} {\ce round} (\emph{type} x) \\
\emph{type} {\ce trunc} (\emph{type} x) }
\indexapi{floor()} \indexapi{ceil()} \indexapi{round()} \indexapi{trunc()}

Various rouinding methods: {\cf floor} returns the highest integer less
than or equal to $x$; {\cf ceil} returns the lowest integer greater than
or equal to $x$; {\cf round} returns the the closest integer to $x$, in
either direction; and {\cf trunc} returns the integer part of $x$
(equivalent to {\cf floor} if $x>0$ and {\cf ceil} if $x<0$).
\apiend

\apiitem{\emph{type} {\ce mod} (\emph{type} a, \emph{type} b) \\
\emph{type} {\ce fmod} (\emph{type} a, \emph{type} b)}
\indexapi{mod()}
\indexapi{fmod()}
Computes $a - b*\mbox{floor}(a/b)$.  The two functions are synonyms.
\apiend

\apiitem{\emph{type} {\ce min} (\emph{type} a, \emph{type} b) \\
\emph{type} {\ce max} (\emph{type} a, \emph{type} b) \\
\emph{type} {\ce clamp} (\emph{type} x, \emph{type} minval, \emph{type} maxval)}
\indexapi{min()} \indexapi{max()} \indexapi{clamp()}
The {\cf min()} and {\cf max()} functions return the minimum or maximum,
respectively, of a list of two or more values.  The {\cf clamp}
function returns

\hspace{2em} {\cf min(max(x,minval),maxval)},

\noindent that is, the value $x$ clamped to the specified range.
\apiend

\apiitem{\emph{type} {\ce mix} (\emph{type} x, \emph{type} y, float alpha)}
\indexapi{mix()}
The {\cf mix} function returns a linear blending :
$ x*(1-\alpha) + y*(\alpha) $
\apiend

\apiitem{float {\ce isnan} (float x) \\
float {\ce isinf} (float x) \\
float {\ce isfinite} (float x)}
\indexapi{isnan()} \indexapi{isinf()} \indexapi{isfinite()}
The {\cf isnan()} function returns 1 if $x$ is a not-a-number (NaN)
value, 0 otherwise.  The {\cf isinf()} function returns 1 if $x$ is an
infinite (Inf or --Inf) value, 0 otherwise.  The {\cf isfinite()}
function returns 1 if $x$ is an ordinary number (neither infinite nor
NaN), 0 otherwise.  \apiend

\begin{annotate}
Should these return int, like the C99 ones?
\end{annotate}

\apiitem{float {\ce erf} (float x) \\
float {\ce erfc} (float x)}
\indexapi{erf()} \indexapi{erfc()}
The {\cf erf()} function returns the error function 
${\mathrm{erf}(x) = \frac{2}{\sqrt{\pi}} \int_0^x e^{-t^2}} dt$.
The {\cf erfc} returns the complementary error function {\cf 1-erf(x)}
(useful in maintaining precision for large values of $x$).
\apiend

\section{Geometric functions}
\label{sec:stdlib:geom}

\apiitem{\emph{ptype} {\ce ptype} (float f) \\
\emph{ptype} {\ce ptype} (float x, float y, float z)}
\indexapi{point()}
\indexapi{vector()}
\indexapi{normal()}

Constructs a point-like value (\emph{ptype} may be any of \point,
\vector, or \normal) from individual \float values.  If constructed
from a single \float, the value will be replicated for $x$, $y$, and $z$.

\apiend

\apiitem{\emph{ptype} {\ce ptype} (string space, f) \\
\emph{ptype} {\ce ptype} (string space, float x, float y, float z)}
\indexapi{point()} \indexapi{vector()} \indexapi{normal()}
Constructs a point-like value (\emph{ptype} may be any of \point,
\vector, or \normal) from individual \float coordinates, relative
to the named coordinate system.  In other words,
\begin{code}
    point (space, x, y, z)
\end{code}
\noindent is equivalent to
\begin{code}
    transform (space, "common", point(x,y,z))
\end{code}
(And similarly for \vector/\normal and {\cf transformv}/{\cf
transformn}, respectively.)
\apiend

\apiitem{float {\ce dot} (vector A, vector B)}
\indexapi{dot()}
Returns the inner product of the two vectors (or normals), i.e., 
$A \cdot B = A_x B_x + A_y B_y + A_z C_z$.
\apiend

\apiitem{float {\ce cross} (vector A, vector B)}
\indexapi{cross()}
Returns the cross product of two vectors (or normals), i.e., 
$A \times B$.
\apiend

\apiitem{float {\ce length} (vector V) \\
float {\ce length} (normal V)}
\indexapi{length()}
Returns the length of a vector or normal.
\apiend

\apiitem{float {\ce distance} (point P0, point P1)}
\indexapi{distance()}
Returns the distance between two points.
\apiend

\apiitem{float {\ce distance} (point P0, point P1, point Q)}
\indexapi{distance()}
Returns the distance from {\cf Q} to the closest point on the line
segment joining {\cf P0} and {\cf P1}.

\begin{annotate}
\QUESTION In RMan this was called `ptlined'.  Do we like that better, or `distance'?
\end{annotate}
\apiend

\apiitem{vector {\ce normalize} (vector V) \\
vector {\ce normalize} (normal V)}
\indexapi{normalize()}
Return a vector in the same direction as $V$ but with length 1,
that is, {\cf V / length(V)} .
\apiend

\apiitem{vector {\ce faceforward} (vector N, vector I, vector Nref) \\
vector {\ce faceforward} (vector N, vector I)}
\indexapi{faceforward()}
If {\cf dot (Nref, I)} $<0$, returns {\cf N}, otherwise returns {\cf -N}.
For the version with only two arguments, {\cf Nref} is implicitly {\cf
Ng}, the true surface normal.  The point of these routines is to
return a version of {\cf N} that faces towards the camera --- in the
direction ``opposite'' of {\cf I}.

To further clarify the situation, here is the implementation of
{\cf faceforward} expressed in \langname:

\begin{code}
vector faceforward (vector N, vector I, vector Nref)
{
    return (I.Nref > 0) ? -N : N;
}

vector faceforward (vector N, vector I)
{
    return faceforward (N, I, Ng);
}
\end{code}
\apiend

\apiitem{vector {\ce reflect} (vector I, vector N)}
\indexapi{reflect()}
For incident vector {\cf I} and surface orientation {\cf N}, returns the
reflection direction {\cf R = I - 2*(N.I)*N}.  Note that {\cf N} must be
normalized (unit length) for this formula to work properly.
\apiend

\apiitem{vector {\ce refract} (vector I, vector N, float eta)}
\indexapi{refract()}
For incident vector {\cf I} and surface orientation {\cf N}, 
returns the refraction direction using Snell's law. The {\cf eta} 
parameter is the ratio of the index of refraction of the volume containing
{\cf I} divided by the index of refraction of the volume being entered.
\apiend

\apiitem {void {\ce fresnel} (vector I, normal N, float eta, \\
 \mbox{\hspace{1in}}   output float Kr, output float Kt,  \\
 \mbox{\hspace{1in}}   output vector R, output float T); }
\indexapi{fresnel()}
According to Snell's law and the Fresnel equations, {\cf fresnel}
computes the reflection and transmission direction vectors {\cf R} and
{\cf T}, respectively, as well as the scaling factors for reflected
and transmitted light, {\cf Kr} and {\cf Kt}.  The {\cf I} parameter
is the normalized incident ray, {\cf N} is the normalized surface normal,
and {\cf eta} is the ratio of refractive index of the medium
containing {\cf I} to that on the opposite side of the surface.
\apiend

\apiitem{point {\ce rotate} (point Q, float angle, point P0, point P1)}
\indexapi{rotate()}
Returns the point computed by rotating point {\cf Q} by {\cf angle} 
radians about the axis that passes from point {\cf P0} to {\cf P1}.
\apiend

transform, transformv, transformn (m, p)
transform, transformv, transformn (s, s, p)
transform, transformv, transformn (s, p)
transform, transformv, transformn (m, m, p)

transformu (s, x)
transformu (s, s, x)




\section{Color functions}
\label{sec:stdlib:color}

\apiitem{color {\ce color} (float f) \\
color {\ce color} (float r, float g, float b)}
\indexapi{color()}
Constructs a \color from individual \float values.  If constructed
from a single \float, the value will be replicated for $r$, $g$, and $b$.
\apiend

\apiitem{color {\ce color} (string colorspace, f) \\
color {\ce color} (string colorspace, float r, float g, float b)}
\indexapi{color()}
Constructs an RGB \color that is equivalent to the individual \float values in a
named color space.  In other words,
\begin{code}
    color (colorspace, r, g, b)
\end{code}
\noindent is equivalent to
\begin{code}
    transformc (colorspace, "rgb", color(r, g, b))
\end{code}
\apiend

\apiitem{float {\ce luminance} (color rgb)}
\indexapi{luminance()}
Returns the linear luminance of the color \emph{rgb}, which is
implemented per the ITU-R standard as $0.2126 R + 0.7152 G + 0.0722 B$.
\apiend

\apiitem{color {\ce transformc} (string fromspace, string tospace, color Cfrom)\\
color {\ce transformc} (string tospace, color Cfrom)}
\indexapi{transformc()}
Transforms color \emph{Cfrom} from color space \emph{fromspace} to
color space \emph{tospace}.  If \emph{fromspace} is not supplied,
it is assumed to be transforming from \rgbspace.

\begin{annotate}
\QUESTION Should this be renamed `ctransform' to match RMan?
\end{annotate}
\apiend


\section{Matrix functions}
\label{sec:stdlib:matrix}

\apiitem{matrix {\ce matrix} (float m00, float m01, float m02, float  m03, \\
\bigspc\spc float m10, float m11, float m12, float m13, \\
\bigspc\spc float m20, float m21, float m22, float m23, \\
\bigspc\spc float m30, float m31, float m32, float m33)}
\indexapi{matrix()}
Constructs a \matrix from 16 individual \float values, in row-major
order.  
\apiend

\apiitem{matrix {\ce matrix} (float f)}
\indexapi{matrix()}
Constructs a \matrix with $f$ in all diagonal components, 0 in all other
components.  In other words, {\cf matrix(1)} is the identity matrix, and
{\cf matrix(f)} is {\cf f*matrix(1)}.
\apiend

\apiitem{matrix {\ce matrix} (string fromspace, float m00, ..., float m33) \\
matrix {\ce matrix} (string fromspace, float f)}
\indexapi{matrix()}
Constructs a \matrix relative to the named space, multiplying it by the
{\cf space}-to-{\cf common} transformation matrix.  

Note that {\cf matrix (space, 1)} returns the 
\emph{space}-to-{\cf common} transformation matrix.
\apiend

\apiitem{float {\ce determinant} (matrix M)}
\indexapi{determinant()}
Computes the determinant of matrix $M$.
\apiend

\apiitem{matrix {\ce transpose} (matrix M)}
\indexapi{transpose()}
Computes the transpose of matrix $M$.
\apiend


\begin{comment}
translate
rotate
scale
\end{comment}


\section{Pattern generation}
\label{sec:stdlib:pattern}

\apiitem{float {\ce step} (float edge, float x)}
\indexapi{step()}
Returns 0 if $x < {\mathit edge}$ and 1 if $x \ge {\mathit edge}$.
\apiend

\apiitem{float {\ce smoothstep} (float edge0, float edge1, float x)}
\indexapi{smoothstep()}
Returns 0 if $x \le {\mathit edge0}$, and 1 if $x \ge {\mathit edge1}$,
and performs a smooth Hermite
interpolation between 0 and 1 when ${\mathit edge0} < x < {\mathit edge1}$.
This is useful in cases where you would want a thresholding function
with a smooth transition.
\apiend

%\section{Noise functions}
%\label{sec:stdlib:noise}
%
%Noise functions (sometimes called ``Perlin noise'')

\apiitem{\emph{type} {\ce noise} (float u) \\
\emph{type} {\ce noise} (float u, float v) \\
\emph{type} {\ce noise} (point p) \\
\emph{type} {\ce noise} (point p, float t)}
\indexapi{noise()}

Returns a continuous, pseudo-random (but repeatable) scalar field
defined on a domain of dimension 1 (\float), 2 (2 \float's), 3 (\point),
or 4 (\point and \float).

The range of \noise is $[0,1]$, its large-scale average is 0.5, it is 
fairly isotropic and nonperiodic (or at least has an extremely large
period), and mostly band-limited to frequences between 0.5 and 1.0.
This makes it ideal to use as a basis function for pattern generation.

The return \emph{type} may be any of \float, \color, \point, \vector, or
\normal, depending on the type of the variable the result is assigned to
(or based on an explicit type cast).  For multi-component return types
(e.g., \point), each component is an uncorrelated \float \noise
function.
\apiend

\apiitem{\emph{type} {\ce snoise} (float u) \\
\emph{type} {\ce snoise} (float u, float v) \\
\emph{type} {\ce snoise} (point p) \\
\emph{type} {\ce snoise} (point p, float t)}
\indexapi{snoise()}
The \snoise function (\emph{signed} noise) is just like
\noise, but scaled so that its range is $[-1,1]$ and its large-scale
average is 0.
\apiend

\apiitem{\emph{type} {\ce pnoise} (float u, float uperiod) \\
\emph{type} {\ce pnoise} (float u, float v, float uperiod, float vperiod) \\
\emph{type} {\ce pnoise} (point p, point pperiod) \\
\emph{type} {\ce pnoise} (point p, float t, point pperiod, float tperiod) 
\smallskip \\
\emph{type} {\ce psnoise} (float u, float uperiod) \\
\emph{type} {\ce psnoise} (float u, float v, float uperiod, float vperiod) \\
\emph{type} {\ce psnoise} (point p, point pperiod) \\
\emph{type} {\ce psnoise} (point p, float t, point pperiod, float tperiod)}
\indexapi{pnoise()}
\indexapi{psnoise()}
The \pnoise function is just like \noise, but is periodic with the given
period.  In other words, {\cf pnoise(x,period) == pnoise(x+period,period)}.
Periods are only meaningful if they are positive integers, so actually
the period is rounded down to the nearest integer, and clamped to a
minimum of 1.

The \psnoise function a signed version of \pnoise, i.e., identical to
\pnoise but having a range of $[-1,1]$ and average value of 0.
\apiend

\apiitem{\emph{type} {\ce cellnoise} (float u) \\
\emph{type} {\ce cellnoise} (float u, float v) \\
\emph{type} {\ce cellnoise} (point p) \\
\emph{type} {\ce cellnoise} (point p, float t)}
\indexapi{cellnoise()}
Returns a discrete pseudo-random (but repeatable) scalar field
defined on a domain of dimension 1 (\float), 2 (2 \float's), 3 (\point),
or 4 (\point and \float).  

The \cellnoise function is constant on $[i,i+1)$ for all
integers $i$, in other words, {\cf cellnoise(x) == cellnoise(floor(x))},
but has a different and uncorrelated value at every integer.  The range
is $[0,1]$, its large-scale average is 0.5, and its values are evenly
distributed over $[0,1]$.

The return \emph{type} may be any of \float, \color, \point, \vector, or
\normal, depending on the type of the variable the result is assigned to
(or based on an explicit type cast).  For multi-component return types
(e.g., \point), each component is an uncorrelated \float \cellnoise
function.
\apiend

\apiitem{\emph{type} {\ce hash} (float u) \\
\emph{type} {\ce hash} (float u, float v) \\
\emph{type} {\ce hash} (point p) \\
\emph{type} {\ce hash} (point p, float t)}
\indexapi{hash()}
Returns a deterministic, repeatable \emph{hash} of the 1-, 2-, 3-, or
4-D coordinates.  The return values will be evenly distributed on
$[0,1]$ and be be completely repeatable when passed the same coordinates
again, yet will be uncorrellated to hashes of any other positions
(including nearby points).  This is like having a random value indexed
spatially, but that will be repeatable from frame to frame of an
animation (provided its input is \emph{precisely} identical).

The return \emph{type} may be any of \float, \color, \point, \vector, or
\normal, depending on the type of the variable the result is assigned to
(or based on an explicit type cast).  For multi-component return types
(e.g., \point), each component is an uncorrelated \float {\cf hash}
function.
\apiend

\apiitem{\emph{type} {\ce random} ()}
\indexapi{random()}
Returns a pseudorandom value uniformly distributed on $[0,1]$.  Use with
extreme caution --- the results are not necessarily repeatable from
frame to frame.

The return \emph{type} may be any of \float, \color, \point, \vector, or
\normal, depending on the type of the variable the result is assigned to
(or based on an explicit type cast).  For multi-component return types
(e.g., \point), each component is an uncorrelated \float {\cf random()}
function.
\apiend

\apiitem{\emph{type} {\ce spline} (string basis, float x, \emph{type} $\mathtt{y}_0$, \emph{type} $\mathtt{y}_1$, ... \emph{type} $\mathtt{y}_{n-1}$)\\
\emph{type} {\ce spline} (string basis, float x, \emph{type} y[])}
\indexapi{spline()}

As $x$ varies from 0 to 1, {\cf spline} returns the value of a cubic
interpolation of uniformly-spaced knots $y_0$...$y_{n-1}$, or
$y[0]$...$y[n-1]$ for the array version of the call (where $n$ is the
length of the array).  The input value $x$ will be clamped to lie
on $[0,1]$.  The \emph{type} may be any of \float, \color,
\point, \vector, or \normal; for multi-component types (e.g. \color),
each component will be interpolated separately.

The type of interpolation is specified by the \emph{basis} name,
\emph{basis} parameter, which may be any of: \qkw{catmull-rom},
\qkw{bezier}, \qkw{bspline}, \qkw{hermite}, or \qkw{linear}.  Some basis
types require particular numbers of knot values -- Bezier splines
require $4n+3$ values, Hermite splines require $4n+2$ values,
Catmull-Rom and linear splines may use any number of values $n\ge 4$.
To maintain consistency with the other spline types, linear splines will
ignore the first and last data value, interpolating piecewise-linearly
between $y_1$ and $y_{n-2}$.
\apiend

\apiitem{float {\ce inversespline} (string basis, float v, float $\mathtt{y}_0$, ... float $\mathtt{y}_{n-1}$)\\
float {\ce inversespline} (string basis, float v, float y[])}
\indexapi{inversespline()}

Computes the \emph{inverse} of the {\cf spline()} function --- i.e., returns
the value $x$ for which {\cf spline (basis, x, y...)} would return value
$v$.  Results are undefined if the knots do not specifiy a monotonic
(only increasing or only decreasing) spline.
\apiend



\section{Derivatives and area operators}
\label{sec:stdlib:derivs}

\apiitem{float {\ce deltau} (float x), {\ce deltav} (float x) \\
vector {\ce deltau} (point x), {\ce deltav} (point x) \\
vector {\ce deltau} (vector x), {\ce deltav} (vector x)\\
color {\ce deltau} (color x), {\ce deltav} (color x)}
\indexapi{deltau()} \indexapi{deltav()}
Compute differentials of the argument $x$, i.e., the approximate change
in $x$ between adjacent shading samples in each of two principal
directions.  Depending on the renderer implementation, those directions
may be $x$ and $y$ on the imaging plane, may be parametric $u$ and $v$
on the surface, or something else.  Do not confuse the ``u'' and ``v''
in the name with a guarantee that it is tied to the surface parameters.
\apiend

\apiitem{float {\ce Du} (float x), {\ce Dv} (float x) \\
vector {\ce Du} (point x), {\ce Dv} (point x) \\
vector {\ce Du} (vector x), {\ce Dv} (vector x)\\
color {\ce Du} (color x), {\ce Dv} (color x)}
\indexapi{Du()} \indexapi{Dv()}
Compute an approximation to the derivatives of $x$ with respect to
each of two principal directions.  Depending on the renderer
implementation, those directions may be $x$ and $y$ on the imaging
plane, may be parametric $u$ and $v$ on the surface, or something else.
Do not confuse the ``u'' and ``v'' in the name with a guarantee that it
is tied to the surface parameters.
\apiend

\begin{annotate}
\QUESTION I'm uncomfortable with these definitions.  Does this make
sense at all in a renderer-agnostic way?  Is it horribly confusing to
call it ``u'' and ``v'' in a ray tracer that's probably computing
derivatives with respect to screen x and y?  If we called it deltax,
deltay, Dx, Dy, would that be confusing in a renderer that really was
Reyes-like and shading in parameter space?

For that matter, is there any legitimate use for differentials or
derivatives along the principal directions, or is the only really
useful quantity filterwidth()?
\end{annotate}

\apiitem{float {\ce filterwidth} (float x) \\
vector {\ce filterwidth} (point x) \\
vector {\ce filterwidth} (vector x)}
\indexapi{filterwidth()}
Compute differentials of the argument $x$, i.e., the approximate change
in $x$ between adjacent shading samples.
\apiend


\apiitem{float {\ce area} (point p)}
\indexapi{area()}
Returns the differential area of position $p$ corresponding to this
shading sample.  If $p$ is the actual surface position \P, then 
{\ce area(P)} will return the surface area of the section of the
surface that is ``covered'' by this shading sample.
\apiend

\apiitem{vector {\ce calculatenormal} (point p)}
\indexapi{calculatenormal()}
Returns a vector perpendicular to the surface that is defined by point
$p$ (as $p$ is computed at all points on the currently-shading surface),
taking into account surface orientation.
\apiend

\apiitem{float {\ce aastep} (float edge, float s) \\
float {\ce aastep} (float edge, float s, float ds) \\
float {\ce aastep} (float edge, float s, float dedge, float ds)
\smallskip \\
float {\ce aastep} (float edge, float s, string filter) \\
float {\ce aastep} (float edge, float s, float ds, string filter) \\
float {\ce aastep} (float edge, float s, float dedge, float ds, string filter)}
\indexapi{aastep()}
Computes an antialiased step function, similar to {\cf step(edge,s)} but
filtering the edge to take into account how rapidly {\cf s} and {\cf edge}
are changing over the surface.  If the differentials {\cf ds} and/or
{\cf dedge} are not passed explicitly, they will be automatically 
computed (using {\cf aastep()}).

The optional {\cf filter} parameter specifies which filter should be
used: \qkw{catmull-rom}, \qkw{box}, \qkw{triangle}, or \qkw{gaussian}.
If no {\cf filter} parameter is supplied, a Catmull-Rom filter will be
used.

\begin{annotate}
\QUESTION This is similar to RenderMan's ``filterstep'', but not quite.
Is it better to call it ``filterstep'' even if it doesn't exactly match,
or should we stick to the changed name to match the changed functionality?
\end{annotate}
\apiend


\section{Displacement functions}
\label{sec:stdlib:displace}

displace
bump

\section{String functions}
\label{sec:stdlib:string}

\apiitem{void {\ce printf} (string fmt, ...)}
\indexapi{printf()}

Much as in C, {\cf printf()} takes a format string {\cf fmt} and an argument list, and
prints the resulting formatted string to the console.  

Where the
fmt contains the characters {\cf \%f}, {\cf \%c}, {\cf \%p}, {\cf
\%m}, and {\cf \%s}, {\cf printf} will substitute arguments, in order,
from the argument list (assuming that the argument types are \float,
\color, point-like, \matrix, and {\cf string}, respectively).  In addition,
{\cf \%d} and {\cf \%i} will also print \float's, truncating and
printing them as if they were integers.
\apiend

\apiitem{string {\ce format} (string fmt, ...)}
\indexapi{format()}
The {\cf format} function works similarly to {\cf printf}, except that
instead of printing the results, it returns the formatted text as a
\string.
\apiend

\apiitem{void {\ce error} (string fmt, ...)}
\indexapi{error()}
The {\cf error()} function works similarly to {\cf printf}, but the
results will be printed as a renderer error message, possibly including
information about the name of the shader and the object being shaded,
and other diagnostic information.
\apiend

\apiitem{void {\ce fprintf} (string filename, string fmt, ...)}
\indexapi{fprintf()}
The {\cf fprintf()} function works similarly to {\cf printf}, but rather
than printing to the default text output stream, the results will be
concatentated onto the end of the text file named by {\cf filename}.
\apiend

\apiitem{string {\ce concat} (string s1, ..., string sN)}
\indexapi{concat()}
Concatenates a list of strings, returning the aggregate string.
\apiend

\begin{annotate}
I'm still gathering requirements for regular expression matching
functions.
\end{annotate}

\apiitem{int {\ce regex_match} (string subject, string regex)}
\indexapi{match()}
Returns 1 if the entire string \emph{subject} matches a standard POSIX 
regular expression \emph{regex}, 0 if it does not.
\apiend

\apiitem{string {\ce substr} (string s, int start, int length)\\
string {\ce substr} (string s, int start)}
\indexapi{substr()}
Return at most \emph{length} characters from \emph{s}, starting with the
character indexed by \emph{start} (beginning with 0).  If \emph{length}
is omitted, return the rest of \emph{s}, starting with \emph{start}.  If
\emph{start} is negative, it counts backwards from the end of the string
(for example, {\cf substr(s,-1)} returns just the last character of
\emph{s}).
\apiend


\begin{annotate}
Should we look at Pystring and add useful things here?
\end{annotate}


\section{Texture}
\label{sec:stdlib:texture}

texture
texture3d
spatialdbsave
spatialdbquery

\section{Light and Shadows}
\label{sec:stdlib:light}

environment
shadow
ambient
diffuse
specular
specularbrdf
indirect
occlusion
subsurface


\section{Renderer state and message passing}
\label{sec:stdlib:state}

getattibute
setmessage
getmessage
gettextureinfo
raylevel
isshadowray
isindirectray


\section{Miscellaneous}
\label{sec:stdlib:misc}

arraylength
exit



\chapter{Formal Language Syntax}
\label{chap:formalsyntax}

BNF

keywords

reserved words






%%%%%%%%%%%%%%%%%%%%%%%%%%%%%%%%%%%%%%%%%%%%%%%%%%%%%%%%%%%%%%%%%%%

\part{Appendices}
\begin{appendix}

%\include{typedesc}
%\chapter{Building Unnamed Shading Language}

%\include{header}
%\include{glossary}
\end{appendix}

\backmatter

%\bibliographystyle{alpha}	%% Select for [FB95]
\bibliographystyle{apalike}    %% Select for (Foo and Bar, 1995)
%\addcontentsline{toc}{chapter}{Bibliography}
%\bibliography{mybib}

\addcontentsline{toc}{chapter}{Index}
\printindex

\end{document}


% Canonical figure
%\begin{figure}[ht]
%\noindent
%\includegraphics[width=5in]{Figures/bredow/foo} 
%\caption{Caption
%\label{fig:foo}}
%\end{figure}
